
\subsection*{Question 1}
Est-ce que la chaleur échangée entre un système et son extérieur est une variable d'état du système. Justifier votre réponse.


\begin{tcolorbox}
    Non, la chaleur échangée par le système avec son extérieur 
    n'est pas une variable d'était car $\delta Q$ n'est pas une différentielle exacte $\Rightarrow \oint \delta Q \neq 0$
    Ce n'est donc pas une variable d'état.\\
    
    On peut également le montrer par un contre exemple, si la chaleur échangée était une variable d'était, alors elle serait indépendante de la transformation.
    Hors on calcul que la chaleur échangée par une transformation adiabatique et une transformation isotherme n'est pas la même.
\end{tcolorbox}

\subsection*{Question 2}

Donner la définition d'un processus \textit{adiabatique}. Donner un exemple d'un processus adiabatique et réversible ainsi qu'un exemple de processus adiabatique et irréversible.

\begin{tcolorbox}
    Un processus est adiabatique quand $\delta Q = 0$ tout au long de la transformation.\\

    Un exemple de processus adiabatique réversible est la compression d'un piston dont les parois sont parfaitement isolées et sans frottements.\\

    Pour rendre la transformation adiabatique irréversible, il suffit d'ajouter des frottements.
\end{tcolorbox}

\subsection*{Question 3}

Donner la définition mathématique ainsi que la signification physique de $c_p$ et $c_v$.

\begin{tcolorbox}
    Les définitions mathématiques des chaleurs spécifiques sont les suivantes
    \begin{align*}
        c_p = \left(\frac{\partial h}{\partial T}\right)_p & &  c_v = \left(\frac{\partial u}{\partial T}\right)_v
    \end{align*}
    
    $c_p$ représente la quantité de chaleur qu'il faut fournir à 1kg de matière pour la faire monter de 1°K à pression constante.\\
    $c_v$ représente la quantité de chaleur qu'il faut fournir à 1kg de matière pour la faire monter de 1°K à volume constant.
    
\end{tcolorbox}

\subsection*{Question 4}

Donner la définition de l'enthalpie libre $F$ de Helmoltz ainsi que de l'enthalpie libre $G$ de Gibbs. Écrivez l'équation de Gibbs sous la forme différentielle de $F$ et $G$


\begin{tcolorbox}
    On a 
    \begin{align*}
        F = U - TS & & G = H - TS\\
        dF = dU - TdS - SdT & & dG = dH - TdS - SdT
    \end{align*}

    Les formules de Gibbs sont les suivantes : 
    \begin{align*}
        TdS = dU + pdV & & TdS = dH - Vdp
    \end{align*}

    En injectant les différentielles de $F$ et $G$ dans les formules de Gibbs, 
    on trouve
    \begin{align*}
        TdS &=  dU + pdV\\
        dU - SdT - dF &= dU + pdV\\
        -SdT &= dF + pdV \\
        dG - dH + TdS &= dF + pdV
    \end{align*}

    Aller plus loin ? 
\end{tcolorbox}

\subsection*{Question 5}
Dérivez l'équation (1.18), page 9 des notes du chapitre 1, $\alpha = p \beta K$

\begin{tcolorbox}
    On a une matrice qui doit posséder un nullspace non trivial $\Rightarrow$ On annule sont déterminant
\end{tcolorbox}


\subsection*{Question 6}
Démontrer que les expressions suivantes sont valables pour toutes les espèces,
\begin{align*}
    \left(\frac{\partial T}{\partial P}\right)_s = \left(\frac{\partial v}{\partial s}\right)_P & & \left(\frac{\partial s}{\partial P}\right)_T = -\left(\frac{\partial v}{\partial T}\right)_P
\end{align*}

\begin{tcolorbox}
    On utilise le théorème de Schwartz.

    \begin{align*}
        F = U - TS & \Rightarrow dF = -pdv - SdT\\
        G = H - TS & \Rightarrow dG = vdp - SdT\\
    \end{align*}
    Sur base de ces différentielles, on trouve que 
    \begin{align*}
        \left( \frac{\partial F}{\partial V}\right)_T = -p & & \left( \frac{\partial F}{\partial T}\right)_v = -s & & \left( \frac{\partial G}{\partial p}\right)_T = v & & \left( \frac{\partial G}{\partial T}\right)_p = -s
    \end{align*}

    Par le théorème de Schwartz, 
    \begin{align*}
        \frac{\partial^2 F }{\partial T \partial v} &= \frac{\partial^2 F }{\partial v \partial T}\\
        \left( \frac{\partial p}{\partial T}\right)_v &= \left( \frac{\partial s}{\partial v}\right)_T
    \end{align*}
    D'où on trouve que 
    \begin{equation*}
        \left(\frac{\partial T}{\partial P}\right)_s = \left(\frac{\partial v}{\partial s}\right)_P
    \end{equation*}

    On fait le même résonnement avec G pour obtenir la seconde égalité.
\end{tcolorbox}


\subsection*{Question 7}
Dériver l'équation suivante : 
\begin{equation*}
    c_p - c_v = \alpha \beta p v T
\end{equation*}

\begin{tcolorbox}
    Réponse au point 4 des notes de cours, chapitre 1.
\end{tcolorbox}

\subsection*{Question 8}

\begin{tcolorbox}
    Réponse au point 4 des cnotes de cours, chapitre 1.
\end{tcolorbox}